\documentclass[acmsmall, nonacm, screen]{acmart}

\usepackage[utf8]{inputenc}
\usepackage[T1]{fontenc}
\usepackage{listings}
\usepackage{caption}
\usepackage{subcaption}
\usepackage{float}
\usepackage{xcolor}
\usepackage{stmaryrd}
\usepackage{anyfontsize}

\newif\ifdraft\drafttrue

\definecolor{green}{HTML}{298a33}
\definecolor{orange}{HTML}{995d02}

\newcommand{\outline}[1]{
  \ifdraft
  {\color{red}{#1}}
  \fi
}

\newcommand\doverline[1]{%
  \setbox0=\hbox{$\overline{#1}$}%
  \ht0=\dimexpr\ht0-.30ex\relax% CHANGE .15 TO AFFECT SPACING
  \overline{\copy0}%
}

\newcommand{\ifThenElse}[3]{\textsf{\color{ACMDarkBlue}if}~#1~\textsf{\color{ACMDarkBlue}then}~#2~\textsf{\color{ACMDarkBlue}else}~#3}
\newcommand{\caseOf}[2]{\textsf{\color{ACMDarkBlue} case}~#1~\textsf{\color{ACMDarkBlue}of}~\{~#2~\}}
\newcommand{\letIn}[3]{\textsf{\color{ACMDarkBlue}let}~#1 = #2~\textsf{\color{ACMDarkBlue}in}~#3}
\newcommand{\shift}[2]{\textsf{\color{ACMDarkBlue}shift}~#1~\textsf{\color{ACMDarkBlue}in}~#2}
\newcommand{\callcc}[2]{\textsf{\color{ACMDarkBlue}call/cc}~#1~\textsf{\color{ACMDarkBlue}in}~#2}
\newcommand{\reset}[1]{\langle #1 \rangle}
\newcommand{\lambdaE}[2]{\lambda #1.\, #2}
\newcommand{\just}[1]{\textsf{Just}~#1}
\newcommand{\nothing}{\textsf{Nothing}}
\newcommand{\map}[3]{\textsf{map}^{\textsf{#1}}~#2~#3}
\newcommand{\unit}[2]{\textsf{unit}^{\textsf{#1}}~#2}
\newcommand{\join}[2]{\textsf{join}^{\textsf{#1}}~#2}
\newcommand{\cps}[1]{\mathcal{C}\llbracket #1 \rrbracket}
\newcommand{\cpsm}[1]{\mathcal{C}'\llbracket #1 \rrbracket}
\newcommand{\cpsmc}[1]{\mathcal{C}''\llbracket #1 \rrbracket}
\newcommand{\denote}[1]{\mathcal{E}\llbracket #1 \rrbracket}
\newcommand{\stringE}[1]{\textsf{\color{green} ``#1''}}
\newcommand{\quoteE}[1]{{\color{orange} \ulcorner #1 \urcorner}}
\newcommand{\unquoteE}[1]{{\color{black} \llparenthesis #1 \rrparenthesis }}

\lstset{ %
  backgroundcolor=\color{white},
  commentstyle=\color{ACMGreen},
  keywordstyle=\color{ACMDarkBlue},
  stringstyle=\color{ACMPurple},
  basicstyle=\ttfamily
}

\lstdefinestyle{hs}{
  language=Haskell
}

\lstdefinestyle{rkt}{
  language=lisp,
  deletekeywords={get},
  morekeywords={define},
  literate=*{(}{{\textcolor{gray}{(}}}{1}
    {)}{{\textcolor{gray}{)}}}{1}
}

\AtBeginDocument{%
  \providecommand\BibTeX{{%
    \normalfont B\kern-0.5em{\scshape i\kern-0.25em b}\kern-0.8em\TeX}}}


\acmBooktitle{}

\begin{document}

\title{Delimited Continuations and Monads}
\subtitle{Understanding from First Principles}
\titlenote{
  This report was compiled for part of the University of Pennsylvania's WPE-II Exam. The
  accompanying talk is available on the author's website.
}

\author{Harrison Goldstein}
\email{hgo@seas.upenn.edu}
\orcid{0000−0001−9631−1169}
\affiliation{%
  \institution{University of Pennsylvania}
  \city{Philadelphia, PA}
  \country{USA}
}

\renewcommand{\shortauthors}{Goldstein}

\begin{abstract}
  In 1990, two programming abstractions were introduced independently: {\em delimited
  continuations} and {\em monads}. These ideas quickly gained popularity, and over the course of
  30 years they have permeated programming languages literature. Modern researchers often assume
  knowledge of these abstractions when writing papers, which streamlines paper exposition but
  makes it difficult for beginners to get started.
  
  This report presents both delimited continuations and monads from first principles by following
  the papers that originally popularized the ideas. First I discuss {\em Abstracting Control} by
  \citet{danvy1990abstracting}, and then I explore {\em Comprehending Monads} by
  \citet{wadler1990comprehending}. Though the two abstractions initially seem to have little to
  do with one-another, delimited continuations and monads actually have a lot in common: they can
  implement many of the same design patterns, and their meta-theories are surprisingly
  compatible. I leverage these connections to develop intuition and provide context for future
  reading.
\end{abstract}

\maketitle

\section{Introduction} \label{sec:introduction}
Abstractions are one of the most powerful tools in a computer scientist's toolbox. They separate
high-level ideas from low-level details, highlighting the novel and the interesting. Sometimes
communities become so used to a particular abstraction that it becomes ingrained: members of the
community naturally think in terms of the abstraction and therefore use it without explanation.
Broadly, I would argue that this is good---a powerful common language makes it easier to
communicate ideas---but it does make it more difficult for outsiders and new researchers to
understand a body of work. Occasionally we must take a step back and explain where these
abstractions come from and how they work.

One such ingrained abstraction in the programming languages community is {\em delimited
continuations}. These were presented in {\em Abstracting Control} by
\citet{danvy1990abstracting}, and they are one of the many tools used to manipulate program
continuations.

A {\em continuation} is a first-class representation of the ``rest'' of a computation. For
example, when evaluating the sub-expression ``$2$'' in the expression
\[ \letIn{i}{84}{\letIn{j}{2}{i / j}}, \]
the remaining {\em context} is
\[ \letIn{i}{84}{\letIn{j}{\_}{i / j}}, \]
and the continuation, $k$, is
\[ k = \lambdaE{x}{\letIn{i}{84}{\letIn{j}{x}{i / j}}}. \]
The continuation captures the context as a function, which can be manipulated as a first-class
value. Danvy and Filinski present two operators for manipulating continuations: {\em shift}
(written ``$\shift{k}{e}$'') and {\em reset} (written ``$\reset{e}$''). Reset delimits the
context, and shift packages that context as a continuation. For example, can use shift and reset
to extract the continuation from above:
\begin{align*}
  & \reset{\letIn{i}{84}{\letIn{j}{(\shift{k}{k~2})}{i / j}}} \\
  \rightarrow\ & \letIn{k}{(\lambdaE{x}{\letIn{i}{84}{\letIn{j}{x}{i / j}}})}{k~2}
\end{align*}
We replaced the expression $2$ with $\shift{k}{k~2}$, which says ``capture the context as $k$ and
then apply that to $2$,'' and we wrapped the whole computation in $\reset{\cdot}$ to delimit the
context. The result is the same as the original expression; the continuation is simply made
explicit. In this example the shift body just applies $k$, but the shift body can be anything. In
particular, if $k$ is not used at all
\[ \reset{\letIn{i}{84}{\letIn{j}{(\shift{k}{\textsf{Nothing}})}{i / j}}} \]
the result is essentially an {\em exception}: the context is thrown away and the result of the
computation is just \textsf{Nothing}. This flexibility makes delimited continuations powerful but
also notoriously difficult for newcomers to reason about.

Perhaps even more ingrained in programming languages research is the {\em monad}. Monads are
almost comically ingrained, to the point where people jokingly say ``a monad is just a monoid in
the category of endofunctors''~\cite{iry_2009,mac2013categories}---a sentence which is
incomprehensible even to most community insiders. The Internet is filled with monad tutorials
that compare the mathematical structure to burritos (tasty, but unhelpful) and other ``more
accessible'' concepts, but few provide useful explanations. To avoid falling into this trap of
explaining monads badly, I follow one of the earliest and clearest explanations of monads for
programming, {\em Comprehending Monads} by \citet{wadler1990comprehending}.

Wadler presents monads using an analogy to list comprehensions like
\[
  [ (x,\,y) \mid x \leftarrow [1,\, 2] ;\ y \leftarrow [3,\, 4] ],
\]
which evaluates to the Cartesian product of the lists ``$[1,\, 2]$'' and ``$[3,\, 4]$.'' Wadler
observed that comprehension syntax can be used to program with more than just lists. For example,
\[
  [ i / j \mid i \leftarrow [84]^{\textsf{Maybe}} ;\ j \leftarrow \ifThenElse{n = 0}{\textsf{Nothing}}{[n]^{\textsf{Maybe}}} ]^{\textsf{Maybe}}
\]
binds $i$ to the value $84$, and then checks if $n = 0$. If it is, the whole expression results
in \textsf{Nothing}, otherwise $j$ is bound to $n$ and we compute $i / j$. The list comprehension
the syntax provided a way to sequence lists, while this comprehension provides a way to sequence
computations that might {\em fail}. The superscript \textsf{Maybe} is actually a type
constructor defined as
\[
  \textsf{type Maybe}~\alpha = \textsf{Nothing} \mid \textsf{Just}~\alpha.
\]
\textsf{Maybe} is a monad, and it is often used to implement exceptions in languages that do not
have them as a first-class feature. There are dozens of monads that programmers use in practice,
each of which gives a different interpretation of comprehension syntax.

Both delimited continuations and monads are interesting on their own, but why present them
together? It is not immediately clear that manipulating continuations has anything to do with
interpreting comprehension syntax; one would be forgiven for assuming that these ideas are
largely orthogonal. But we have already seen that delimited continuations and monads can both be
used to simulate exceptions, and it turns out that there is significantly more overlap. Design
patterns like state, nondeterminism, and more can be implemented using both delimited
continuations and monads. Furthermore, continuations actually form a monad, and Wadler shows that
this fact can be exploited to recover results from Danvy and Filinski.

In this report I make three contributions:
\begin{itemize}
  \item I explain delimited continuations and monads by following two influential papers: {\em
  Abstracting Control} by Danvy and Filinski (\S~\ref{sec:danvy}) and {\em Comprehending Monads}
  by Wadler (\S~\ref{sec:wadler}).
  \item I develop intuition for programming with delimited continuations by implementing a number
  of the design patterns that Wadler presents as monads. This solidifies the close relationship
  between the two language features (\S~\ref{sec:patterns}).
  \item I present the {\em continuation monad} and show that, by performing a monad-agnostic
  transformation given by Wadler, we can recover the ``extended continuation passing style''
  transformation presented by Danvy and Filinski (\S~\ref{sec:contmonad}). This small extension
  of Wadler's original result exemplifies the importance of monads as a theoretical tool.
\end{itemize}
I conclude with some remarks on the broader context of these abstractions (\S~\ref{sec:conclusion}).

\subsection*{Unified Notation} \label{sec:notation}
My two source papers differ slightly on notations and conventions. I have chosen a reasonable
middle-ground and use one unified set of conventions to discuss both papers. Throughout this
report, assume that programs are written in a lambda calculus given by the grammar:
\begin{align*}
  e \Coloneqq &\ x \mid \lambdaE{x}{e} \mid e_1~e_2 \\
            | &\ \textsf{\color{ACMDarkBlue} true} \mid \textsf{\color{ACMDarkBlue} false} \mid \ifThenElse{e_1}{e_2}{e_3} \mid n \mid e_1 + e_2 \\
            | &\ (e_1,\, e_2) \mid \textsf{\color{ACMDarkBlue} fst}~e \mid \textsf{\color{ACMDarkBlue} snd}~e
\end{align*}
The core calculus is just variables, $x$, lambda abstractions $\lambdaE{x}{e}$, and function
application $e_1~e_2$. Danvy and Filinski use Booleans and natural numbers to illustrate
delimited continuations, and Wadler uses pairs to illustrate monads---I will include these
features as-needed. Since we are almost exclusively concerned with the dynamic semantics of these
programs, I will not formalize a type system.

\section{{\em Abstracting Control}} \label{sec:danvy}
Before explaining delimited continuations, I should explain some background on continuations and
first-class continuation operators. Continuations first appeared as meta-theoretical
tools~\cite{strachey2000continuations}, but before long they were introduced into concrete
languages via the {\em continuation-passing style} or CPS translation. Here is a call-by-value
CPS translation, $\cps{\cdot}$, for our basic lambda calculus with Booleans and if-statements:
\begin{align*}
  \cps{x} &= \lambdaE{\kappa}{\kappa~x} \\
  \cps{\lambdaE{x}{e}} &= \lambdaE{\kappa}{\kappa~(\lambdaE{x}{\cps{e}})} \\
  \cps{e_1~e_2} &= \lambdaE{\kappa}{\cps{e_1}~(\lambdaE{f}{\cps{e_2}~(\lambdaE{x}{f~x~\kappa})})} \\
  \cps{\textsf{\color{ACMDarkBlue}true}} &= \lambdaE{\kappa}{\kappa~\textsf{\color{ACMDarkBlue} true}} \\
  \cps{\ifThenElse{e_1}{e_2}{e_3}} &= \lambdaE{\kappa}{\cps{e_1}~(\lambdaE{b}{\ifThenElse{b}{\cps{e_2}~\kappa}{\cps{e_3}~\kappa}})}
\end{align*}
This translation makes all continuations explicit, and translated terms do not ``return'' in the
traditional sense. When these terms finish computing, they pass the resulting value to a
continuation, $\kappa$, which is provided as an argument. A translated term can be run by passing
the trivial continuation: $\cps{e}~(\lambdaE{x}{x})$. \outline{Say more here}

The CPS translation has been used extensively in compilers and interpreters, where it gives
fine-grained control over evaluation. This is because the translation makes evaluation order
explicit---this particular CPS translation is clearly call-by-value, since in the rule for
application,
\[ \cps{e_1~e_2} = \lambdaE{\kappa}{\cps{e_1}~(\lambdaE{f}{\cps{e_2}~(\lambdaE{x}{f~x~\kappa})})} \]
the argument $e_2$ is evaluated before $f$ is applied.

As CPS translations became more popular, researchers began to consider first-class abstractions
for working with continuations. The most famous of these early abstractions, ``call with current
continuation'' (or ``call/cc'')~\cite{haynes1984continuations}, can be added as an operator in
our source language and implemented as an extension to the CPS translation:
\[ \cps{\callcc{k}{e}} = (\lambdaE{\kappa}{\cps{e}~\kappa})[k \mapsto \lambdaE{x}{\lambdaE{\kappa'}{\kappa~x}}], \]
where ``$e[x \mapsto v]$'' means $e$ with $v$ substituted for free instances of $x$.

The call/cc operator is extremely powerful. The expression $\callcc{k}{e}$ captures the current
program context as $k$ and then runs $e$, allowing the programmer to arbitrarily pause and resume
evaluation as they see fit. For example,
\[ \callcc{k}{1} \]
aborts the computation, returning the value $1$, and
\[ \callcc{k}{(\textsf{\color{ACMDarkBlue} print}~\stringE{foo};\ k~10)} \]
pauses the computation, prints ``foo'', and then resumes computation with the value 10.
Unfortunately, many have said that call/cc is {\em too} powerful. Consider the popular Yin-Yang
Puzzle:
\begin{align*}
& \textsf{\color{ACMDarkBlue} let}\ \textsf{yin}\ =\ (\lambdaE{c}{(\textsf{\color{ACMDarkBlue} print}~\stringE{@};\ c)})~(\callcc{k}{k})~\textsf{\color{ACMDarkBlue} in} \\
& \textsf{\color{ACMDarkBlue} let}\ \textsf{yang}\ =\ (\lambdaE{c}{(\textsf{\color{ACMDarkBlue} print}~\stringE{*};\ c)})~(\callcc{k}{k})~\textsf{\color{ACMDarkBlue} in} \\
& \textsf{yin}~\textsf{yang}
\end{align*}
What does this confusing mess of continuations do? Apparently it counts. This program prints
increasing numbers in a unary representation,
\[ \stringE{@*@**@***@****@*****...}, \]
but it is extremely difficult to understand exactly why that is.

\citet{felleisen1988theory} reigned in the power of call/cc with operators that he called
``prompt'' and ``control''. These allowed programmers to delimit the effects of call/cc with {\em
scopes}. Unfortunately, Felleisen's scopes were dynamic, and his approach did not admit a
straightforward translation into a standard lambda calculus. This was the impetus for Danvy and
Filinski's statically delimited continuations.

\subsection{Delimited Continuations}
In {\em Abstracting Control}, Danvy and Filinski introduce the ``shift'' and ``reset'' operations
mentioned in Section \ref{sec:introduction}, which provide a more usable alternative to call/cc.
\[
  e \Coloneqq \cdots \mid \shift{k}{e} \mid \reset{e}
\]
These operators can also be interpreted via a modified CPS translation that Danvy and Filinski
call {\em extended continuation-passing style} (ECPS):
\begin{align*}
  \cps{\shift{k}{e}} &= (\lambdaE{\kappa}{\cps{e}~\textsf{id}})[k \mapsto \lambdaE{x}{\lambdaE{\kappa'}{\kappa'~(\kappa~x)}}] \\
  \cps{\reset{e}} &= \lambdaE{\kappa}{\kappa~(\cps{e}~\textsf{id})}
\end{align*}
The translation for $\reset{e}$ translates $e$ and extracts its final value by applying it to
\textsf{id}. This means that any continuation manipulation done by $e$ does not leak outside of
the scope of the reset, so we say that reset ``delimits'' the continuation. The translation for
shift gives $e$ access to its continuation ($\kappa$) via the variable $k$.

Together, shift and reset provide an intuitive interface for working with continuations. For
example, take the following expression and its evaluation:
\begin{align*}
& 1 + \reset{10 + \shift{k}{k~(k~100)}} \Rightarrow \\
& 1 + (10 + (10 + 100)) \Rightarrow \\
& 121
\end{align*}
The continuation ``$k = \lambdaE{x}{10 + x}$'' captured by the shift operator, since ``$10 +
\shift{k}{\dots}$'' is within a context delimited by reset. The outer ``$1 +$'' is not captured
because it is outside of the reset. The function $k$ is applied twice to $100$, resulting in
$121$, and then we add $1$, resulting in $121$.

Danvy and Filinski point out that the ECPS translation, with rules for shift and reset, is no
longer truly in continuation-passing style. In particular, the rule for shift means that the
resulting program might not enforce strict call-by-value semantics, the translation of shift
introduces a redex $\kappa'~(\kappa~x)$. This could be solved by translating the result a second
time, but we can use this as an opportunity to take a step back and look at another way of
defining the semantics of our calculus with shift and reset.

In addition to the ECPS translation, Danvy and Filinski give a denotational translation. Whereas
the ECPS translation is purely syntactic, this transformation is {\em semantic} and maps every
program to a mathematical object in a given {\em domain}. In this case the domain would contain
Booleans (since we have \textsf{\color{ACMDarkBlue}true} and \textsf{\color{ACMDarkBlue}false} as
base values) and computable functions. This is common practice in programming languages
literature and hearkens back to the origin of continuations~\cite{strachey2000continuations}.
Assuming $\textsf{Ans}$ is a suitable domain of final answers, we define:
\begin{center}
  \begin{tabular}{llr}
    $\rho \in \textsf{Env}$ & $=$ & $\textsf{Var} \rightharpoonup \textsf{Val}$ \\
    $\gamma \in \textsf{MCont}$ & $=$ & $\textsf{Val} \to \textsf{Ans}$ \\
    $\kappa \in \textsf{Cont}$ & $=$ & $\textsf{Val} \to \textsf{MCont} \to \textsf{Ans}$ \\
    $\mathcal{E}$ & $:$ & $\textsf{Exp} \to \textsf{Env} \to \textsf{Cont} \to \textsf{MCont} \to \textsf{Ans}$
  \end{tabular}
\end{center}
The variable environment, $\rho$, maps variables to domain values, $\gamma$ is a
``meta''-continuation that captures the continuation {\em outside} of the closest enclosing
reset, and $\kappa$ is the continuation that we are used to (inside of the closest reset). The
denotational semantics is given by the equations:
\begin{align*}
  \denote{x}~\rho~\kappa~\gamma &= \kappa~(\rho[x])~\gamma \\
  \denote{\lambdaE{x}{e}}~\rho~\kappa~\gamma &= \kappa~(\lambdaE{v}{\denote{e}~(\rho[x \mapsto v])})~\gamma \\
  \denote{e_1~e_2}~\rho~\kappa~\gamma &=
    \denote{e_1}~\rho~(\lambdaE{f}{\denote{e_2}~\rho~(\lambdaE{x}{f~x~\kappa})})~\gamma \\
  \denote{\textsf{\color{ACMDarkBlue}true}}~\rho~\kappa~\gamma &= \kappa~\textsf{\color{ACMDarkBlue} true}~\gamma \\
  \denote{\ifThenElse{e_1}{e_2}{e_3}}~\rho~\kappa~\gamma &= 
    \denote{e_1}~\rho~(\lambdaE{b}{\ifThenElse{b}{\denote{e_2}~\rho~\kappa}{\denote{e_3}~\rho~\kappa}})~\gamma \\
  \denote{\shift{k}{e}}~\rho~\kappa~\gamma &=
    \denote{e}~(\rho[k \mapsto \lambdaE{x}{\lambdaE{\kappa'}{\lambdaE{\gamma'}{\kappa~x~(\lambdaE{w}{\kappa'~w~\gamma'})}}}])~(\lambdaE{x}{\lambdaE{\gamma''}{\gamma''~x}})~\gamma \\
  \denote{\reset{e}}~\rho~\kappa~\gamma &= \denote{e}~\rho~(\lambdaE{x}{\lambdaE{\gamma'}{\gamma'~x}})~(\lambdaE{x}{\kappa~x~\gamma})
\end{align*}
Remember that the constructions on the right-hand side are part of a meta-language and represent
mathematical objects rather than concrete syntax. Rather than build syntactic functions that take
continuations as arguments our continuations are embedded in the meta-theory.

When reading these rules, it is usually safe to ignore the $\gamma$ arguments. For all rules
other than the ones for shift and reset, $\gamma$ can be $\eta$-reduced away (since
$\lambdaE{x}{f~x}$ is equivalent to $f$). The first five equations look a lot like the CPS
translation, just lifted to a semantic domain. The meta-continuations are used in the shift and
reset rules to capture the outer continuation as a computable function.

All of this is important for theoretical work on this language, but the syntactic ECPS
transformation is more useful for concrete implementations. Next, we will return to the ECPS
translation and explore ways that it can be made more efficient.

\subsection{Metacircular Interpreters}
A {\em metacircular interpreter} is a powerful tool for expressing a language's
semantics~\cite{reynolds1972definitional}. In the case of our shift-reset language, our
metacircular interpreter will look a lot like the ECPS translation, but it will do a bit more
work by translating constructs an executable meta-language, rather than generating syntax
directly. Danvy and Filinski use this approach to build a more efficient ECPS translation.

For this interpreter, our meta-language is the same lambda calculus we have been working with,
but without shift or reset. We use a simple {\em quoting} mechanism to specify which parts of the
output are executable code and which are concrete syntax. The code in black is the executable
interpreter, while the $\quoteE{\text{quoted}}$ code is concrete syntax output. We use
$\unquoteE{\text{unquote brackets}}$ to splice computations into quoted segments.
\begin{align*}
  \cpsm{x} &= \lambdaE{\kappa}{\kappa~x} \\
  \cpsm{\lambdaE{x}{e}} &=
    \lambdaE{\kappa}{\kappa~\quoteE{\lambdaE{\unquoteE{x}}{\lambdaE{k}{\unquoteE{\cpsm{e}~(\lambdaE{a}{\quoteE{k~\unquoteE{a}}})}}}}} \\
  \cpsm{e_1~e_2} &= \lambdaE{\kappa}{\cpsm{e_1}~(\lambdaE{f}{\cpsm{e_2}~(\lambdaE{x}{\quoteE{\unquoteE{f}~\unquoteE{x}~(\lambdaE{t}{\unquoteE{\kappa~\quoteE{t}}})}})})} \\
  \cpsm{\textsf{\color{ACMDarkBlue}true}} &= \lambdaE{\kappa}{\kappa~\textsf{\color{ACMDarkBlue} true}} \\
  \cpsm{\ifThenElse{e_1}{e_2}{e_3}} &= \lambdaE{\kappa}{\cpsm{e_1}~(\quoteE{\lambdaE{b}{\ifThenElse{b}{\unquoteE{\cpsm{e_2}~\kappa}}{\unquoteE{\cpsm{e_3}~\kappa}}}})} \\
  \cpsm{\shift{k}{e}} &= (\lambdaE{\kappa}{\cpsm{e}~\textsf{id}})[k \mapsto \quoteE{\lambdaE{x}{\lambdaE{\kappa'}{\kappa'~\unquoteE{\kappa~\quoteE{x}}}}}] \\
  \cpsm{\reset{e}} &= \lambdaE{\kappa}{\kappa~(\cpsm{e}~\textsf{id})}
\end{align*}
The final ECPS result is given by ``$\cpsm{e}~\textsf{id}$.''

Similar to the denotational semantics, the continuations are all meta-functions; there are no
$\kappa$s in the resulting programs. The difference is that the meta-functions are executed to
produce concrete syntax, rather than interpreted as abstract functions. For example, the rule for
if-statements will produce a concrete term of the form $\lambdaE{b}{\ifThenElse{\_}{\_}{\_}}$,
rather than a mathematical function. Representing continuations at the meta-level drastically
reduces the size of the resulting programs---there are no unnecessary redexes in the output.

It turns out that we can do even better if we include delimited continuation operators in our
meta-language. This final ECPS translation, 
\begin{align*}
  \cpsmc{x} &= x \\
  \cpsmc{\lambdaE{x}{e}} &= \quoteE{\lambdaE{\unquoteE{x}}{\lambdaE{\kappa}{\unquoteE{\reset{\quoteE{\kappa~\unquoteE{\cpsmc{e}}}}}}}} \\
  \cpsmc{e_1~e_2} &= \shift{\kappa}{\quoteE{\unquoteE{\cpsmc{e_1}}~\unquoteE{\cpsmc{e_2}}~(\lambdaE{t}{\unquoteE{\kappa~\quoteE{t}}})}} \\
  \cpsmc{\textsf{\color{ACMDarkBlue}true}} &= \textsf{\color{ACMDarkBlue}true} \\
  \cpsmc{\ifThenElse{e_1}{e_2}{e_3}} &= \shift{\kappa}{\quoteE{\ifThenElse{\unquoteE{\cpsmc{e_1}}}{\unquoteE{\reset{\kappa~\cpsmc{e_2}}}}{\unquoteE{\reset{\kappa~\cpsmc{e_3}}}}}} \\
  \cpsmc{\shift{k}{e}} &= \shift{\kappa}{\reset{\cpsmc{e}}[k \mapsto \quoteE{\lambdaE{x}{\lambdaE{\kappa'}{\kappa'~\unquoteE{\kappa~\quoteE{x}}}}}]} \\
  \cpsmc{\reset{e}} &= \reset{\cpsmc{e}}
\end{align*}
also avoids unnecessary $\eta$-expansion (i.e., it always produces $f$ instead of
$\lambdaE{x}{f~x}$), resulting in a pleasingly compact final representation.

This final interpreter is essentially written in the same language that it interprets---this is
why metacircular interpreters are called ``meta'' and it does pose a bootstrapping problem.
Luckily, we already have both the ECPS translation and our denotational semantics as descriptions
of this meta-language. The efficient interpreter can be written in this meta-language and then
that interpreter can itself be ECPS transformed into a standard lambda calculus.

\subsection{Use Case: Nondeterministic Programming} \label{sec:danvy:nondet}
These different ways of giving meaning to shift and reset are important, but often the best way
to understand a concepts is to work with examples. We will look at a bunch of examples in Section
\ref{sec:patterns}, but for now we will look at one that Danvy and Filinski mention:
nondeterministic computations.

Danvy and Filinski start by defining operators \textsf{fail} and \textsf{flip}, which form the
basis of nondeterministic computation:
\begin{align*}
\textsf{fail}~() &= \shift{k}{\stringE{failure}} \\
\textsf{flip}~() &= \shift{k}{(k~\textsf{\color{ACMDarkBlue}true};\ k~\textsf{\color{ACMDarkBlue}false};\ \textsf{fail}~())}
\end{align*}
We can think of \textsf{flip} as flipping a coin (with sides \textsf{\color{ACMDarkBlue}true} and
\textsf{\color{ACMDarkBlue}false}) and \textsf{fail} as signaling a ``dead-end'' outcome.

Concretely \textsf{fail} operation acts as an exception. It throws away the continuation and
returns the string $\stringE{failure}$. The \textsf{flip} operation calls its continuation on
{\em both} \textsf{\color{ACMDarkBlue}true} and \textsf{\color{ACMDarkBlue}false}, which
simulates a nondeterministic choice between the two options. This works because the result of the
nondeterministic computation is actually a sequence of values---all of the possible outcomes.
When we call \textsf{flip}, the possible options are \textsf{\color{ACMDarkBlue}true} and
\textsf{\color{ACMDarkBlue}false}, so we continue with both. If later in the computation one of
those values is undesirable, we can call \textsf{fail}.

The \textsf{choice} operation builds on \textsf{flip} to simulate a nondeterministic choice of
positive integers less than a given value.
\[
  \textsf{choice}~n = \ifThenElse{n < 1}{\textsf{fail~()}}{\ifThenElse{\textsf{flip}~()}{\textsf{choice}~(n - 1)}{n}}
\]
Calling ``$\textsf{choice}~2$'' will continue with $2$, then $1$, and then $\stringE{failure}$.

Finally, Danvy and Filinski implement the function
\begin{align*}
\textsf{triple}~n~s =&  \\
& \textsf{\color{ACMDarkBlue}let}~i = \textsf{choice}~n~\textsf{\color{ACMDarkBlue}in} \\
& \textsf{\color{ACMDarkBlue}let}~j = \textsf{choice}~(i - 1)~\textsf{\color{ACMDarkBlue}in} \\
& \textsf{\color{ACMDarkBlue}let}~k = \textsf{choice}~(j - 1)~\textsf{\color{ACMDarkBlue}in} \\
& \ifThenElse{i + j + k = s}{(i,\, j,\, k)}{\textsf{fail}~()},
\end{align*}
which finds all triples of distinct positive integers that sum to a given integer $s$. The
complexity is pushed into \textsf{choice}, so this function is nice and readable. Evaluating the
expression ``$\reset{\textsf{\color{ACMDarkBlue}print}~(\textsf{triple}~9~15)}$'' prints all
triples of integers up to $9$ that sum to $15$.

The \textsf{triple} example is mostly a toy based on an early benchmark, but these primitives can
be practically useful. In the paper, Danvy and Filinski show an evaluator for {\em
nondeterministic finite automata} (NFAs), which is straightforward to implement thanks to
operators like \textsf{flip} and \textsf{fail}. The implementation is not especially instructive,
so I will not repeat it here.

% \subsection{Generalizing to More Contexts}

% The final contribution in {\em Abstracting Control} is an idea for a {\em family} of shift and
% reset operators that operate on an indexed family of program contexts. They introduce operators
% \begin{center}
%   $\textsf{\color{ACMDarkBlue}shift}_i~k~\textsf{\color{ACMDarkBlue}in}~e$ \hspace{5mm}
%   and \hspace{5mm} $\reset{e}_i$
% \end{center}
% which access and delimit the $i$th enclosing context respectively. The goal here is to be able to
% associate different simulated effects (e.g. error handling and nondeterminism) with different
% indexed operators; without such a mechanism the effects do not compose well.

% These operators can also be given meaning via the ECPS translation, but with a bit more hassle. A
% program that uses indices up to $n$ will need to be translated $n + 1$ times. The translations
% are defined by mutual induction on term structure (as before) and on operator indices.
% \begin{align*}
%   \cps{\textsf{\color{ACMDarkBlue}shift}_0~k~\textsf{\color{ACMDarkBlue}in}~e} &= e[k \mapsto \textsf{id}] \\
%   \cps{\textsf{\color{ACMDarkBlue}shift}_{n + 1}~k~\textsf{\color{ACMDarkBlue}in}~e} &= 
%     \lambdaE{\kappa}{\textsf{\color{ACMDarkBlue}shift}_{n}~k'~\textsf{\color{ACMDarkBlue}in}~(\cps{e}~\textsf{id})[k \mapsto \lambdaE{x}{\lambdaE{\kappa'}{\kappa'~\reset{k'~(\kappa~x)}_n}}]} \\
%   \cps{\reset{e}_0} &= e \\
%   \cps{\reset{e}_{n + 1}} &= \lambdaE{\kappa}{\kappa~\reset{\cps{e}~\textsf{id}}_n}
% \end{align*}
% See Danvy and Filinski's paper for more of the meta-theory for these constructs.

We will revisit delimited continuation operators to look at more examples in Section
\ref{sec:patterns}, but for now I will take a break from continuations and introduce monads.

\section{{\em Comprehending Monads}} \label{sec:wadler}
As mentioned in Section \ref{sec:introduction}, monads as popular as they are poorly understood.
They were originally introduced to the programming languages literature by
\citet{moggi1991notions}, who was working in {\em category theory}. Luckily, monads can be
understood without learning a new branch of mathematics, and {\em Comprehending Monads} by
\citet{wadler1990comprehending} is a great place to start.

A monad is essentially just a data structure (e.g., \textsf{List} or \textsf{Maybe}) with certain
associated operations. First, a monad must have an operation \textsf{map} that applies a function
``under'' the type constructor. For example,
\[ \map{List}{f}{\overline{x}} \]
applies $f$ to all elements of $\overline{x}$ and 
\[ \map{Maybe}{f}{\overline{x}} \]
applies $f$ to the value contained in $\overline{x}$, if one exists. (I adopt Wadler's convention
of writing monadic values---values wrapped in the monad's type constructor---with a bar over the
variable name.) Map must obey two laws:
\begin{align*}
  \textsf{map}~\textsf{id} &= \textsf{id} \\
  \textsf{map}~(g \circ f) &= \textsf{map}~g \circ \textsf{map}~f
\end{align*}
The first law says that mapping the identity function over a structure does nothing---this is
necessary to make sure that \textsf{map} does not change the structure itself, only the elements
in the structure. The second law says that \textsf{map} is well behaved with respect to function
composition.

A data structure with a valid \textsf{map} operation is called a {\em functor}. In order for a
functor to be a monad, it needs two more operations: \textsf{unit} and \textsf{join}. Applying
\textsf{unit} to a value injects that value into the monad structure. In the case of
\textsf{List},
\[ \unit{List}{x} = \textsf{singleton}~x = [x], \]
which is the simplest way to inject a value into a list. Likewise,
\[ \unit{Maybe}{x} = \textsf{Just}~x. \]


The \textsf{join} operation works on ``doubled up'' instances of the monad. It takes a doubled
value like
\begin{center}
  ``$[[1, 2], [3, 4]]$''\hspace{5mm}or\hspace{5mm}``$\textsf{Just}~(\textsf{Just}~5)$''
\end{center}
and flattens it down to a single application of the monad like
\begin{center}
  ``$[1, 2, 3, 4]$''\hspace{5mm}or\hspace{5mm}``$\textsf{Just}~5$''.
\end{center}
To be precise:
\begin{center}
  \begin{tabular}{lll}
    $\join{List}{\doverline{x}}$ & $=$ & $\textsf{flatten}~\doverline{x}$ \\
    $\join{Maybe}{\doverline{x}}$ & $=$ & $\caseOf{\doverline{x}}{\textsf{Just}~(\textsf{Just}~x) \to \textsf{Just}~x;\ \_ \to \textsf{Nothing}}$
  \end{tabular}
\end{center}

These operations must obey certain laws in order for the structure to truly be a monad:
\begin{align}
  \textsf{map}~f \circ \textsf{unit} &= \textsf{unit} \circ f \\
  \textsf{map}~f \circ \textsf{join} &= \textsf{join} \circ \textsf{map}~(\textsf{map}~f) \\
  \textsf{join} \circ \textsf{unit} &= \textsf{id} \\
  \textsf{join} \circ \textsf{map}~\textsf{unit} &= \textsf{id} \\
  \textsf{join} \circ \textsf{join} &= \textsf{join} \circ \textsf{map}~\textsf{join}
\end{align}
Law (1) and (2) enforce that \textsf{unit} and \textsf{join} are ``well-behaved'' with respect to
\textsf{map}. In particular, (1) is a strong statement that says that \textsf{unit} cannot {\em
do} very much---it must commute with an arbitrary function $f$ (modulo the monad structure). Law
(3) says that applying \textsf{unit} to a monadic value and then \textsf{join}-ing is a no-op.
Law (4) says the same as (3) except we apply \textsf{unit} {\em under} the monad using
\textsf{map}. Finally, law (5) says that for a triple-stacked monad (e.g., $[[[1]]]$), either way
of flattening it to a single monad (e.g., $[1]$) is the same.

Importantly, the monad laws require that \textsf{unit} and \textsf{join} are the {\em simplest}
ways to do their respective tasks. For example, suppose we tried to implement
\[ \unit{List}{x} = [x,\, x,\, x,\, x,\, x]. \]
Technically this does inject a value into a list, but it is not the simplest way to do so and
\[
  (\textsf{join}^{\textsf{List}} \circ \textsf{unit}^{\textsf{List}})~[3] = [3,\, 3,\, 3,\, 3,\, 3] \neq \textsf{id}~[3]. \\
\]

The \textsf{join} operation can be used to define special kind of function composition called
{\em Kleisli composition}. Given a functor $F$ and two functions $f: \beta \to F~\gamma$ and $g:
\alpha \to F~\beta$, we can write \[ f \circ_F g = \textsf{join}^F \circ \textsf{map}^F~f \circ g
\] that composes them together while properly keeping track of $F$. This is useful in programming
because it means that monadic operations like $f$ and $g$ can be built up individually and
composed together to build larger programs. A very similar construction can also yield the
``\textsf{bind}'' (or ``\textsf{>>=}'') definition of monads that is more common in modern
programming languages. Though these alternative views of monads are probably more useful for
programming, they are slightly harder to work with for theory, so I will continue to use Wadler's
notation.

All of this background on monads and their operations is important, but in practice using these
operations directly does not make for particularly elegant code. In the next section, I explore
Wadler's use of {\em comprehensions} to make programming with monads much more intuitive.

\subsection{Comprehensions}
List comprehensions are based on set-builder notation (e.g. $\{n + 1 \mid n \in \mathbb{N}\}$)
and were first introduced to programming by the SETL language in the
1960s~\cite{schwartz2012programming}. Since then, list comprehensions have become ubiquitous,
appearing in many modern languages including Racket, Python, and even C++ (with some abuse of
operator overloading). Comprehensions are a convenient way of building lists from other lists,
for example a construction like
\[ [(x,\, y) \mid x \leftarrow \overline{x};\ y \leftarrow \overline{y}]^{\textsf{List}} \]
computes the Cartesian product of $\overline{x}$ and $\overline{y}$. The equivalent set-builder
construction looks like
\[ \{(x,\, y) \mid x \in X \wedge y \in Y\}. \]

Wadler starts off by adding a formal syntax for list comprehensions to a standard lambda calculus:
\begin{align*}
 e &\Coloneqq \cdots \mid [e \mid q] \\
 q &\Coloneqq \Lambda \mid x \leftarrow e \mid (q_1;\ q_2)
\end{align*}
The symbol $q$ stands for {\em qualifiers}, which can be empty (denoted $\Lambda$), a binder for
a variable $x$, or a composition of two qualifiers. List comprehension syntax is defined by the
following rules:
\begin{align*}
  [e \mid \Lambda] &= \textsf{singleton}~e \\
  [e_1 \mid x \leftarrow e_2] &= \textsf{map}~(\lambdaE{x}{e_1})~e_2 \\
  [e \mid (q_1;\ q_2)] &= \textsf{flatten}~[[e \mid q_2] \mid q_1]
\end{align*}
This notation can be a bit heavy, so Wadler usually writes ``$[e]$'' instead of ``$[e \mid
\Lambda]$,'' and he proves that qualifier composition is associative so he can write ``$[e \mid
q_1;\ q_2]$'' instead of $[e \mid (q_1;\ q_2)]$.

{\em Comprehending Monads} centers around the observation that lists are not the only structure
for which comprehensions make sense. Since \textsf{singleton} is $\textsf{unit}^{\textsf{List}}$
and \textsf{flatten} is $\textsf{join}^{\textsf{List}}$, we can try to re-interpret
comprehensions for any monad. Wadler observed that every monad \textsf{M} has an associated monad
comprehension given by the following rules:
\begin{align*}
  [e \mid \Lambda]^{\textsf{M}} &= \unit{M}{e} \\
  [e_1 \mid x \leftarrow e_2]^{\textsf{M}} &= \map{M}{(\lambdaE{x}{e_1})}{e_2} \\
  [e \mid (q_1;\ q_2)]^{\textsf{M}} &= \join{M}{[[e \mid q_2]^{\textsf{M}} \mid q_1]^{\textsf{M}}} \\
\end{align*}
This is how we arrive at the \textsf{Maybe} comprehension from Section \ref{sec:introduction},
\[
  [ i / j \mid i \leftarrow [84]^{\textsf{Maybe}} ;\ j \leftarrow \ifThenElse{n = 0}{\textsf{Nothing}}{[n]^{\textsf{Maybe}}} ]^{\textsf{Maybe}}
\]
which cleanly handles the fact that the expression binding $j$ might fail, without introducing
significant control-flow overhead.

Monad comprehensions were the first step towards the ``do--notation'' that is popular in Haskell.
\[ \textsf{\color{ACMDarkBlue} do}\ \{\ i \leftarrow \unit{Maybe}{84};\ j \leftarrow
(\ifThenElse{n = 0}{\nothing}{\unit{Maybe}{n}});\ \unit{Maybe}{i/j}\ \}, \] The distinctions
between comprehension syntax and do--notation are mostly cosmetic. When using do--notation,
programmers write ``$\unit{Maybe}{w}$'' (or more idiomatically ``$\textsf{return}~e$'') instead
of $[e]^{\textsf{M}}$. Also rather than put a return value at the front of the comprehension,
do--blocks evaluate to their final expression. For the remainder of this paper I will stick to
comprehension syntax, but programming-focused monad examples usually use something closer to
do--notation.

\subsection{Examples of Monads} \label{sec:monad-examples}
Wadler presents a whole zoo of monads in his paper; exploring those structures provides valuable
intuition for how monads work and what they can do.

\subsubsection{List and Maybe}
These first two monads have been running examples in this section, so I will not spend much time
on them here. Both the \textsf{List} and \textsf{Maybe} types are defined as informal algebraic
data types, similar to how they would be defined in a language like Haskell. The \textsf{List}
monad gives rise to standard list comprehensions, and the \textsf{Maybe} monad can be used to chain
operations that might potentially fail.
\begin{center}
  \framebox[\textwidth]{
  \begin{tabular}{lll}
    $\textsf{type List}$~$\alpha$ & $=$ & $\textsf{Nil} \mid \textsf{Cons}~\alpha~(\textsf{List}~\alpha)$ \\
    $\map{List}{f}{\overline{x}}$ & $=$ & $\caseOf{\overline{x}}{\textsf{Nil} \to \textsf{Nil};\ \textsf{Cons}~y~\overline{y} \to \textsf{Cons}~(f~y)~(\map{List}{f}{\overline{y}})}$ \\
    $\unit{List}{x}$ & $=$ & $\textsf{singleton}~x$ \\
    $\join{List}{\doverline{x}}$ & $=$ & $\textsf{flatten}~\doverline{x}$ \\
    \\
    $\textsf{type Maybe}$~$\alpha$ & $=$ & $\textsf{Nothing} \mid \textsf{Just}~\alpha$ \\
    $\map{Maybe}{f}{\overline{x}}$ & $=$ & $\caseOf{\overline{x}}{\textsf{Nothing} \to \textsf{Nothing};\ \textsf{Just}~y \to \textsf{Just}~(f~y)}$ \\
    $\unit{Maybe}{x}$ & $=$ & $\textsf{Just}~x$ \\
    $\join{Maybe}{\doverline{x}}$ & $=$ & $\caseOf{\doverline{x}}{\textsf{Just}~(\textsf{Just}~x) \to \textsf{Just}~x;\ \_ \to \textsf{Nothing}}$
  \end{tabular}
  }
\end{center}
\vfill

\subsubsection{Identity and Strictness}
The simplest possible monad is an identity---it does nothing. This monad is almost not worth
mentioning, but it does have one cute use-case: comprehension syntax for the identity monad can
be used instead of \textsf{\color{ACMDarkBlue} let} binding. Rather than write
\begin{center}
  $\letIn{x}{e_1}{\letIn{y}{e_2}{e_3}}$ \hspace{5mm} we can write \hspace{5mm} $[e_3 \mid x \leftarrow e_1;\ y \leftarrow e_2]^{\textsf{Id}}$.
\end{center}
There is no real reason to prefer this syntax, so I will continue to use
$\textsf{\color{ACMDarkBlue} let}$, but it is a fun observation.

A slight variation on the identity monad, the {\em strictness} monad, does actually have some
interesting uses. Assuming a call-by-name base language, the strictness monad would provide
control over evaluation order by altering the definition of $\textsf{map}$ to force the
evaluation of the argument $\overline{x}$. Statements in a \textsf{Str} comprehension would
essentially use call-by-value semantics.
\begin{center}
  \framebox[\textwidth]{
  \begin{tabular}{lll}
    $\textsf{type Id}$~$\alpha$ & $=$ & $\alpha$ \\
    $\map{Id}{f}{\overline{x}}$ & $=$ & $f~\overline{x}$ \\
    $\unit{Id}{x}$ & $=$ & $x$ \\
    $\join{Id}{\doverline{x}}$ & $=$ & $\doverline{x}$ \\
    \\
    $\textsf{type Str}$~$\alpha$ & $=$ & $\alpha$ \\
    $\map{Str}{f}{\overline{x}}$ & $=$ & $\ifThenElse{\overline{x} \neq \bot}{f~\overline{x}}{\bot}$ \\
    $\unit{Str}{x}$ & $=$ & $x$ \\
    $\join{Str}{\doverline{x}}$ & $=$ & $\doverline{x}$
  \end{tabular}
  }
\end{center}
\vfill

\subsubsection{State}
The \textsf{State} monad makes enables code that {\em looks} stateful, even though there are no
actual references under the hood. The state monad is defined above, assuming some state type
$\sigma$. As with many monads, the real magic comes from auxiliary operations:
\begin{center}
  \begin{tabular}{lll}
    $\textsf{get}$ & $=$ & $\lambdaE{s}{(s,\, s)}$ \\
    $\textsf{put}~s'$ & $=$ & $\lambdaE{s}{(x,\, s')}$
  \end{tabular}
\end{center}
These can be used to write the expression
\[
  [k \mid i \leftarrow \textsf{get};\ \_ \leftarrow \textsf{put}~(i + 3);\ j \leftarrow
  \textsf{get};\ \_ \leftarrow \textsf{put}~(j * 7);\ k \leftarrow \textsf{get}]^{\textsf{State}}
\]
which simulates a stateful computation; it can be run to compute the value $42$ (technically
$(42,\, 42)$).
\outline{could say more}
\begin{center}
  \framebox[\textwidth]{
  \begin{tabular}{lll}
    $\textsf{type State}$~$\alpha$ & $=$ & $\sigma \to (\alpha,\, \sigma)$ \\
    $\map{State}{f}{\overline{x}}$ & $=$ & $\lambdaE{s}{\letIn{(x,\, s')}{\overline{x}~s}{(f~x,\, s')}}$ \\
    $\unit{State}{x}$ & $=$ & $\lambdaE{s}{(x,\, s)}$ \\
    $\join{State}{\doverline{x}}$ & $=$ & $\lambdaE{s}{\letIn{(\overline{x},\, s')}{\doverline{x}~s}{\overline{x}~s'}}$
  \end{tabular}
  }
\end{center}
\vfill

\subsubsection{Reader} (Also known as {\em State Reader}.)
The \textsf{Reader} monad is a simplification of the \textsf{State} monad that does not allow the
state to be modified. Despite its relative simplicity, the \textsf{Reader} monad is quite
powerful. After defining the auxiliary operation,
\begin{center}
  \begin{tabular}{lll}
    $\textsf{ask}$ & $=$ & $\lambdaE{r}{r}$,
  \end{tabular}
\end{center}
\textsf{Reader} can be used to keep track of configuration values, environment information, and
any other static value that would otherwise be cumbersome to pass around explicitly. Readers are
also safer and more flexible than global variables or constants, since they are implemented as
functions and thus have a clearly delimited scope.
\begin{center}
  \framebox[\textwidth]{
  \begin{tabular}{lll}
    $\textsf{type Reader}$~$\alpha$ & $=$ & $\rho \to \alpha$ \\
    $\map{Reader}{f}{\overline{x}}$ & $=$ & $\lambdaE{r}{f~(\overline{x}~r)}$ \\
    $\unit{Reader}{x}$ & $=$ & $\lambdaE{r}{x}$ \\
    $\join{Reader}{\doverline{x}}$ & $=$ & $\lambdaE{r}{(\doverline{x}~r)~r}$
  \end{tabular}
  }
\end{center}
\vfill

\subsubsection{Nondeterminism} (Also known as {\em Set}.)
The nondeterminism monad enables direct-style programming of nondeterministic algorithms. It
avoids the messiness of a random number generator \outline{fix this on rewrite} by keeping track
of all possible values at once. Together with auxiliary operators,
\begin{center}
  \begin{tabular}{lll}
    $\textsf{fail}$ & $=$ & $\varnothing$ \\
    $\textsf{flip}$ & $=$ & $\{\textsf{\color{ACMDarkBlue} true},\, \textsf{\color{ACMDarkBlue} false}\}$,
  \end{tabular}
\end{center}
this monad can be extremely useful. \outline{could say more}
\begin{center}
  \framebox[\textwidth]{
  \begin{tabular}{lll}
    $\textsf{type ND}$~$\alpha$ & $=$ & $2^\alpha$ \\
    $\map{ND}{f}{\overline{x}}$ & $=$ & $\{f~x \mid x \in \overline{x}\}$ \\
    $\unit{ND}{x}$ & $=$ & $\{x\}$ \\
    $\join{ND}{\doverline{x}}$ & $=$ & $\bigcup \doverline{x}$
  \end{tabular}
  }
\end{center}
\vfill

\subsubsection{Parser}
Finally, here is a more complicated monad: \textsf{Parser}. Parsers can be seen as a combination
of two monads: \textsf{List} and \textsf{State}. Intuitively, the \textsf{String} argument is the
parser input, and the result is a list of potential parses, each is made up of a result of type
$\alpha$ and the \textsf{String} that remains after parsing.

There are a number of auxiliary operations that are useful for \textsf{Parser} (often called {\em
parser combinators}), including ``\textsf{satisfy}'' which only parses a character that matches a
given predicate and ``\textsf{many}'' which applies a parser multiple times. An amazing amount of
research has been done on the \textsf{Parser} monad and parser
combinators~\cite{hutton1996monadic, leijen2001parsec}, and it is still an active
area~\cite{willis2020staged}.
\begin{center}
  \framebox[\textwidth]{
  \begin{tabular}{lll}
    $\textsf{type Parser}$~$\alpha$ & $=$ & $\textsf{String} \to \textsf{List}~(\alpha,\, \textsf{String})$ \\
    $\map{Parser}{f}{\overline{x}}$ & $=$ & $\lambdaE{i}{[(f~x,\, i') \mid (x,\, i') \leftarrow \overline{x}~i]^{\textsf{List}}}$ \\
    $\unit{Parser}{x}$ & $=$ & $\lambdaE{i}{[(x, i)]^{\textsf{List}}}$ \\
    $\join{Parser}{\doverline{x}}$ & $=$ & $\lambdaE{i}{[(x,\, i'') \mid (\overline{x},\, i') \leftarrow \doverline{x}~i;\ (x,\ i'') \leftarrow \overline{x}~i']^{\textsf{List}}}$
  \end{tabular}
  }
\end{center}
\vfill

\subsection{Translation}
Since monads can be used to replicate a variety of effects in a pure way, it would be nice if
there was a way to translate an impure program into the equivalent monadic one. Wadler gives two
such translations, one for call-by-value and another for call-by-name. The result of both
translations is a pure program in our lambda calculus extended with comprehensions.

The call-by-value translation ``$e^*$'' lifts a program of type $\alpha \to \beta$ to one of type
$\alpha \to \textsf{M}~\beta$ for some monad \textsf{M}. Intuitively, the translated function
takes a value of type $\alpha$ and returns a {\em computation} of type $\beta$. The monad
captures the effects of that computation. Here is the translation in full:
\begin{align*}
  x^* &= [x]^\textsf{M} \\
  (\lambdaE{x}{e})^* &= [\lambdaE{x}{e^*}]^\textsf{M} \\
  (e_1~e_2)^* &= [y \mid f \leftarrow e_1^*;\ x \leftarrow e_2^*;\ y \leftarrow f~x]^\textsf{M} \\
  (e_1,\, e_2)^* &= [(x, y) \mid x \leftarrow e_1^*;\ y \leftarrow e_2^*]^\textsf{M} \\
  (\textsf{\color{ACMDarkBlue}fst}~e)^* &= [\textsf{\color{ACMDarkBlue}fst}~x \mid x \leftarrow e^*]^\textsf{M}
\end{align*}
Each of these rules has a straightforward computational meaning. For example
\[ (e_1~e_2)^* = [y \mid f \leftarrow e_1^*;\ x \leftarrow e_2^*;\ y \leftarrow f~x]^\textsf{M} \]
says that in order to evaluate a (potentially effectful) application, we evaluate $e_1$ to $f$,
then evaluate $e_2$ to $x$, and then evaluate the application ``$f~x$'', all the while keeping
track of any side effects that are produced.

If our source language was actually effectful, for example with built-in effects
``$\textsf{\color{ACMDarkBlue}get}$'' and ``$\textsf{\color{ACMDarkBlue}put}$'', we could fix
$\textsf{M} = \textsf{State}$ and add translations
\begin{align*}
  \textsf{\color{ACMDarkBlue}get}^* &= [x \mid x \leftarrow \textsf{get}]^\textsf{State} \\
  (\textsf{\color{ACMDarkBlue}put}~e)^* &= [u \mid x \leftarrow e^*;\ u \leftarrow \textsf{put}~x]^\textsf{State}
\end{align*}
where \textsf{get} and \textsf{put} are defined as in Section \ref{sec:monad-examples}. Given a
once-and-for-all proof that the monadic version has the same semantics as the impure version,
analysis could be done on the pure language instead. In addition, this kind of translation might
streamline the implementation of a compiler or interpreter.

% Wadler also gives a call-by-name version of the translation, which lifts a program of type
% \begin{center}
% $\alpha \to \beta$ \hspace{5mm} to one of type \hspace{5mm} $\textsf{M}~\alpha \to \textsf{M}~\beta$.
% \end{center}
% In call-by-name, the function arguments are computations as well. The translation is written
% $e^\dagger$:
% \begin{align*}
%   x^\dagger &= x \\
%   (\lambdaE{x}{e})^\dagger &= [\lambdaE{x}{e^\dagger}]^\textsf{M} \\
%   (e_1~e_2)^\dagger &= [y \mid f \leftarrow e_1^\dagger;\ y \leftarrow f~e_2^\dagger]^\textsf{M} \\
%   (e_1,\, e_2)^\dagger &= [(e_1^\dagger,\, e_2^\dagger)]^\textsf{M} \\
%   (\textsf{\color{ACMDarkBlue}fst}~e)^\dagger &= [y \mid x \leftarrow e^\dagger;\ y \leftarrow \textsf{\color{ACMDarkBlue}fst}~x]^\textsf{M}
% \end{align*}
% This is a more obscure translation, but it is useful when the source language is call-by-name.

\outline{redo} These translations mark the last bit of background that we need; now we can start to explore the
connections between monads and delimited continuations.

\section{Common Design Patterns} \label{sec:patterns}
I said in Section \ref{sec:introduction} that both delimited continuations and monads abstract
away complex control flow and simulate effectful code in a pure language. So far we have
certainly seen that monads can do this, but it is less clear that delimited continuations can do
the same. In particular, we have only seen delimited continuations simulate exceptions and
nondeterminism. In this section I will complete the picture and a few more abstractions \outline{ew} that can
be captured using continuations. I was inspired by a blog post by \citet{xia_2019}, which takes
all of these ideas even further, implementing various monads embedded in the {\em continuation
monad} in Haskell---I will discuss that monad in detail in Section \ref{sec:contmonad}.

Following Xia's presentation, I will present each pattern by defining (1) the primitive
operations that are core to the abstraction and (2) a \textsf{run} function that delimits the
scope of the computation.

\subsubsection{Maybe}
\begin{center}
  \begin{tabular}{lll}
    $\textsf{abort}$ & $=$ & $\shift{k}{\textsf{Nothing}}$ \\
    $\textsf{run}^{\textsf{Maybe}}~c$ & $=$ & $\reset{\textsf{Just}~c}$
  \end{tabular}
\end{center}
We have seen exceptions in the contexts of both continuations and monads already, although
admittedly I have been fairly imprecise when talking about using shift to throw an exception.
This construction faithfully simulates the \textsf{Maybe} monad, in that its computations either
fail and return $\nothing$ or succeed with a value $v$ and return $\just{v}$.

\subsubsection{Identity} \outline{can we do strictness?}
\begin{center}
  \begin{tabular}{lll}
    $\textsf{noop}~x$ & $=$ & $\shift{k}{k~x}$ \\
    $\textsf{run}^{\textsf{Id}}~c$ & $=$ & $\reset{c}$
  \end{tabular}
\end{center}
Identity can be encoded without using continuations at all, of course, but this presentation is
instructive. Here the \textsf{noop} operation shows how to do nothing using delimited
continuations---just call the continuation. All \textsf{run} needs to do is delimit the
continuation scope.

\subsubsection{State}
\begin{center}
  \begin{tabular}{lll}
    $\textsf{get}$ & $=$ & $\shift{k}{\lambdaE{s}{k~s~s}}$ \\
    $\textsf{put}~s'$ & $=$ & $\shift{k}{\lambdaE{s}{k~()~s'}}$ \\
    $\textsf{run}^{\textsf{State}}~c~i$ & $=$ & $\reset{(\lambdaE{v}{\lambdaE{s}{v}})~c}~i$
  \end{tabular}
\end{center}
This example is considerably more involved. As with the state monad, encoding state with
continuations is about capturing state in function parameters and return values. Our
continuations take two parameters: the first is the computation value, and the second is the new
state. The \textsf{get} operation takes in the current state and continues with that state as
both the computation value and the new state. The \textsf{put} operation ignores the current
state and continues with unit as the computation value and $s'$ as the new state. The
\textsf{run} function takes an initial state and passes it in {\em outside} the continuation
scope.

As with the monadic version, these operations can be used to simulate stateful code. For example,
this program
\begin{align*}
\textsf{run}^{\textsf{State}}~(& \textsf{\color{ACMDarkBlue}let}~i = \textsf{get}~\textsf{\color{ACMDarkBlue}in} \\
& \textsf{put}~(i + 3); \\
& \textsf{\color{ACMDarkBlue}let}~j = \textsf{get}~\textsf{\color{ACMDarkBlue}in} \\
& \textsf{put}~(j * 7); \\
& \textsf{get})~3
\end{align*}
returns 42 as one might expect, and does so without references or other built-in state constructs.

\subsubsection{Reader}
\begin{center}
  \begin{tabular}{lll}
    $\textsf{ask}$ & $=$ & $\textsf{get}$ \\
    $\textsf{run}^{\textsf{Reader}}$ & $=$ & $\textsf{run}^{\textsf{State}}$
  \end{tabular}
\end{center}
\textsf{Reader} is strictly simpler than \textsf{State}, but we can use the same definitions
anyway. We just rename \textsf{get} to \textsf{ask} and don't provide a \textsf{put} operation.

\subsubsection{Nondeterminism}
\begin{center}
  \begin{tabular}{lll}
    $\textsf{fail}$ & $=$ & $\shift{k}{k~\varnothing}$ \\
    $\textsf{flip}$ & $=$ & $\shift{k}{k~\textsf{\color{ACMDarkBlue} true} \cup k~\textsf{\color{ACMDarkBlue} false}}$ \\
    $\textsf{run}^{\textsf{ND}}~c$ & $=$ & $\reset{(\lambdaE{v}{\{v\}})~c}$
  \end{tabular}
\end{center}
We already saw one way of encoding nondeterminism using continuations (see Section
\ref{sec:danvy:nondet}), but here is another one that is a bit closer to the monadic
presentation. We again make use of sets to define the primitive operations $\textsf{fail}$ and
$\textsf{flip}$: Our implementation of \textsf{run} lifts computations into singleton sets, and
our operators call the continuation on whatever values should be considered---either nothing in
the case of \textsf{fail} or both $\textsf{\color{ACMDarkBlue} true}$ and
$\textsf{\color{ACMDarkBlue} false}$ in the case of $\textsf{flip}$.

\subsection{Are there more?}
This is just the tip of the iceberg when it comes to monadic design patterns that can be
implemented using delimited continuations---in fact, a few years after {\em Abstracting Control},
\citet{filinski1994representing} published another paper showing that delimited continuations can
represent all ``pure'' monads (i.e., those whose \textsf{map}, \textsf{join}, and \textsf{bind}
operations can be implemented in a pure language). We discuss this result further in Section
\ref{sec:conclusion}.

\section{The Continuation Monad} \label{sec:contmonad}
Up to this point we have viewed delimited continuations and monads in parallel, bouncing back and
forth between the abstractions and showing how they compare. But there is one final insight from
Wadler's paper that I skipped over, and I will use it now to build a bridge between continuations
and monads.

Towards the end of {\em Comprehending Monads}, Wadler presents the {\em continuation monad}.
Given a result type, $\rho$, the continuation monad is defined as:
\begin{center}
  \begin{tabular}{lll}
    $\textsf{type Cont}$~$\alpha$ & $=$ & $(\alpha \to \rho) \to \rho$ \\
    $\map{Cont}{f}{\overline{x}}$ & $=$ & $\lambdaE{\kappa}{\overline{x}~(\lambdaE{x}{\kappa~(f~x)})}$ \\
    $\unit{Cont}{x}$ & $=$ & $\lambdaE{\kappa}{\kappa~x}$ \\
    $\join{Cont}{\doverline{x}}$ & $=$ &
      $\lambdaE{\kappa}{\doverline{x}~(\lambdaE{\overline{x}}{\overline{x}~\kappa})}$
  \end{tabular}
\end{center}
Intuitively, $\textsf{Cont}~\alpha$ is the type of computations that take a continuation
expecting an input $\alpha$ and computing a result in a fixed type $\rho$. The \textsf{map}
operation composes a function $f$ with the continuation, \textsf{unit} simply continues with a
given value, and \textsf{join} untangles nested continuations to enable chained computations.

Programs written using the continuation monad are automatically in a continuation-passing style.
A sequence of operations
\[ [(x, y) \mid x \leftarrow \overline{x};\ y \leftarrow \overline{y}]^{\textsf{Cont}} \]
expands to
\[ \lambdaE{\kappa}{\overline{x}~(\lambdaE{x}{\overline{y}~(\lambdaE{y}{\kappa~(x,\, y)})})}, \]
which is precisely the way a tuple of expressions would be evaluated in CPS.

As with many of the monads before, we can define auxiliary operations that interact in useful ways
with the monad operations. We can define our old friends \textsf{shift} and \textsf{reset}!
\begin{center}
  \begin{tabular}{lll}
    $\textsf{shift}$~$f$ & $=$ & $\lambdaE{\kappa}{f~(\lambdaE{x}{\lambdaE{\kappa'}{\kappa'~(\kappa~x)}})~(\lambdaE{x}{x})}$ \\
    $\textsf{reset}$~$\overline{x}$ & $=$ & $\lambdaE{\kappa}{\kappa~(\overline{x}~(\lambdaE{x}{x}))}$
  \end{tabular}
\end{center}
Whereas in Section \ref{sec:danvy} we built these operations into the language, now they are
simply defined. This means that there is no need to extend any meta-theory---we can interpret these
operations in the same lambda calculus that we worked with in Section \ref{sec:wadler}.

In particular, this means that we can use the \textsf{Cont} monad comprehension to write programs like
\[
  [x + y \mid x \leftarrow [1];\ y \leftarrow \textsf{reset}~[u + v \mid u \leftarrow [10];\ v \leftarrow \textsf{shift}(\lambdaE{k}{[b \mid a \leftarrow k~100;\ b \leftarrow k~a]})]]. \\
\]
which is equivalent to the program
\[ 1 + \reset{10 + \shift{k}{k~(k~100)}}, \]
which we saw when discussing {\em Abstracting Control} (although admittedly the monad version is
a bit more verbose). Critically neither version relies on manually managing continuation
parameters; they are both written in a ``direct'' style.

\subsection{Recovering ECPS}
We can tie everything together by looking back at Wadler's call-by-value translation into a monad
\textsf{M}. Recall that the translation takes a function of type $\alpha \to \beta$ in a language
with true effectful operators, and produces a program of type $\alpha \to \textsf{M}~\beta$ in a
pure language where \textsf{M} captures the effect. Though we have not discussed it this way so
far, shift and reset are actually built-in effects (for some definition of effect) \outline{not
great} so we can use the $(\cdot)^*$ transformation to capture those effects in the \textsf{Cont}
monad.

Concretely, we will use $(\cdot)^*$ to translate programs from Section \ref{sec:danvy} into
programs from Section \ref{sec:wadler} where $\textsf{M} = \textsf{Cont}$. By extending Wadler's
$(\cdot)^*$ translation with rules that turn ``$\shift{k}{e}$'' into
``$\textsf{shift}~(\lambdaE{k}{e})$'' and ``$\reset{e}$'' into ``$\textsf{reset}~e$'', we obtain
exactly the ECPS translation presented by Danvy and Filinski!
\begin{align*}
  x^* &= [x]^\textsf{Cont} &= \lambdaE{\kappa}{\kappa~x} \\
  (\lambdaE{x}{e})^* &= [\lambdaE{x}{e^*}]^\textsf{Cont} &= \lambdaE{\kappa}{\kappa~(\lambdaE{x}{e^*})} \\
  (e_1~e_2)^* &= [y \mid f \leftarrow e_1^*;\ x \leftarrow e_2^*;\ y \leftarrow f~x]^\textsf{Cont} &= \lambdaE{\kappa}{e_1^*~(\lambdaE{f}{e_2^*~(\lambdaE{x}{f~x~\kappa})})} \\
  (e_1,\, e_2)^* &= [(x, y) \mid x \leftarrow e_1^*;\ y \leftarrow e_2^*]^\textsf{Cont} &= \lambdaE{\kappa}{e_1^*~(\lambdaE{x}{e_2^*~(\lambdaE{y}{\kappa (x,\, y)})})} \\
  (\textsf{\color{ACMDarkBlue}fst}~e)^* &= [\textsf{\color{ACMDarkBlue}fst}~x \mid x \leftarrow e^*]^\textsf{Cont} &= \lambdaE{\kappa}{e^*~(\lambdaE{x}{\textsf{\color{ACMDarkBlue} fst}~x)}} \\
  (\shift{k}{e})^* &= \textsf{shift}~(\lambdaE{k}{e^*}) &= (\lambdaE{\kappa}{e^*~\textsf{id}})[k \mapsto \lambdaE{x}{\lambdaE{\kappa'}{\kappa'~(\kappa~x)}}] \\
  \reset{e}^* &= \textsf{reset}~e^* &= \lambdaE{\kappa}{\kappa~(e^*~\textsf{id})}
\end{align*}

This is truly beautiful. It means that the continuation monad entirely captures the semantics of
a language with first-class continuation operators. Shockingly, even though Danvy and Filinski
gave their language no fewer than four semantic interpretations (including ECPS, a denotational
semantics, and two meta-circular interpreters) Wadler managed to find one more way of looking at
continuation semantics.

(Of course, Wadler did not present exactly this result. He observed that the call-by-name
translation results in the unusual call-by-name version of the CPS translation, but since shift
and reset did not exist, he did not complete the ECPS picture. I am sure that the full ECPS--monad
embedding is not novel, but I completed the picture independently.)

\section{Conclusion} \label{sec:conclusion}
\outline{Start to wrap up...}

\citet{filinski1994representing} provides a detailed account of how monadic computations can be
represented using delimited continuations. While my constructions in Section \ref{sec:patterns}
are ad-hoc and dependent on the particular monad in question, Filinski managed to give a
translation that represents monadic patterns with continuations once-and-for-all. Still, the
ad-hoc connections are valuable: they provide more granular intuition than Filinski's
presentation, and they are all entirely pure, while Filinski's formulation relies on a reference
cell.

In recent years there has been increasing focus on {\em algebraic effects}.\outline{cite}
Interestingly, algebraic effects have links to both continuations and monads: they operate via
{\em handlers}, which are often written using explicit continuations, but they isolate effects in
much the same way as monadic abstractions.

\outline{Better ending}

\bibliography{references}
\bibliographystyle{plainnat}

\end{document}
